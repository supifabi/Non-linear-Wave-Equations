\documentclass[12pt,a4paper]{article}
\usepackage[utf8]{inputenc}
\usepackage[T1]{fontenc}
\usepackage{amsfonts}
\usepackage{amsthm}
\usepackage{german}
\usepackage{mathrsfs}
\usepackage{csquotes}
\usepackage{lmodern}
\usepackage{microtype}
\usepackage{amsmath}
\usepackage{amssymb}
\usepackage{mathtools}
\usepackage{basicmath}
\usepackage{bbm}

\usepackage{subfig}

\newtheorem{definition}{Definition}
\newtheorem{bem}[definition]{Remark}
\newtheorem{lemma}[definition]{Lemma}
\newtheorem{thm}[definition]{Theorem}
\newtheorem{bsp}[definition]{Example}
\newtheorem{cor}[definition]{Corollary}

\newcommand{\Rn}{\mathbb{R}^n}

\title{Non-Linear Wave Equations Basics}
\author{Fabienne Klatt}
\date{23.11.22}

\begin{document}

\tableofcontents
\thispagestyle{empty}

\clearpage

Here everything can be written down that helps with understanding the Non-Linear Wave Equations Lecture and solving the exercises.

\section{Norms}

\begin{definition}[Sobolev space]

\end{definition}

\begin{definition}[fractional Sobolev space; source: exercise session 1]
For $s\in\mathbb{R}$ (?) $H^s(\mathbb{R}^n)$ is the completion of $C^{\infty}_0(\mathbb{R}^n)$ with respect to the norm
\begin{equation}
\lVert f \rVert_{H^s(\mathbb{R}^n)}=\lVert (1+\lvert\xi\rvert^2)^{\frac{s}{2}}\mathcal{F}(f)\rVert_{L^2(\mathbb{R}^n)}
\end{equation}
where $(1+\lvert\xi\rvert^2)^{\frac{s}{2}}$ corresponds to a derivative in physical space.
\end{definition}

\section{Functional Analysis Theorems}

\begin{thm}[Arzelà-Ascoli theorem, source: Wiki]
Consider a sequence of functions $(f_n)$ defined on a compact interval. If this sequence is uniformly bounded and uniformly equicontinuous, then there exists a subsequence $(f_{n_k})$ that converges uniformly.
\end{thm}

\begin{definition}[uniformly equicontinuous]
A sequence of functions $(f_n)$ is said to be uniformly equicontinuous if, for every $\varepsilon > 0$, there exists a $\delta > 0$ such that
\begin{equation}
\left|f_{n}(x)-f_{n}(y)\right|<\varepsilon
\end{equation}
whenever $| x-y |$ < $\delta$ for all functions $f_n$ in the sequence.
\end{definition}

This is the version of Arzelà-Ascoli used on example sheet 6, problem*, in a specific norm:

\begin{thm}[Arzelà-Ascoli theorem, source: solution to sheet 5]
Suppose there is a sequence (of solutions) $(\phi^{(i)})$ in $C^{k-1}([0,T]\times \mathbb{R}^n)$ that converges to $\phi$ in a $C^{k-1}([0,T]\times \mathbb{R}^n)$-sense and is uniformly bounded, meaning $\lVert \partial ^k \phi \rVert _{L^{\infty}} \leq A$ ($A$ not dependent on $i$).\\
Then, there exists a subsequence $(\phi^{(i_{\lambda})})$ that converges in $C^{k}([0,T]\times \mathbb{R}^n)$.
\end{thm}

\begin{thm}[Banach-Alaoglu]
\end{thm}

\begin{thm}[Riesz Representation thm]
\end{thm}

\section{Fourier Transformation}

Open Question: What is the advantage of Fourier representation?

\begin{definition}[Schwartz space]
The Schwartz space $\mathcal{S}(\mathbb{R}^n)$ is a vectorspace given by
\begin{equation}
\mathcal{S}(\mathbb{R}^n)=\{f\in C^{\infty}(\mathbb{R}^n)\lvert \forall \alpha,\beta\in \mathbb{N}^n_0, \sup_{x\in\mathbb{R}^n}\lvert x^{\alpha}\partial^{\beta}f(x)\rvert <\infty\}
\end{equation}
equipped with a countable family of semi-norms
\begin{equation}
\lVert f \rVert_{\alpha,\beta} := \sup_{x\in\mathbb{R}^n}\lvert x^{\alpha}\partial^{\beta}f(x)\rvert.
\end{equation}
\end{definition}

You can think of the Schwartz space as functions that decay faster than any polynomial.

\begin{definition}[Fourier transform]
Given a function $f\in \mathcal{S}(\mathbb{R}^n)$, we define the Fourier transform by
\begin{equation}
\mathcal{F}(f)(\xi):=\int_{\Rn} f(x)e^{-2\pi ix\cdot \xi} dx.
\end{equation}
The inverse Fourier transform is given by
\begin{equation}
\mathcal{F}^{-1}(f)(\xi):=\int_{\Rn} f(x)e^{+2\pi ix\cdot \xi} dx.
\end{equation}
\end{definition}

\begin{thm}[properties of Fourier transform and norms; source: exercise session 1]
For $f\in \mathcal{S}(\mathbb{R}^n)$ the following holds:
\begin{enumerate}
\item $\lVert \mathcal{F}(f) \rVert_{L^2(\mathbb{R}^n)} = \lVert f \rVert_{L^2(\mathbb{R}^n)}$
\item $\lVert \mathcal{F}(f) \rVert_{L^{\infty}(\mathbb{R}^n)} = \lVert f \rVert_{L^1(\mathbb{R}^n)}$
\end{enumerate}
\end{thm}

\begin{thm}[properties of the Fourier transform; source: exercise session 1]
For $f\in \mathcal{S}(\mathbb{R}^n)$ the following properties hold:
\begin{enumerate}
\item $\mathcal{F}(\partial_if(x))=2\pi i \xi_i \mathcal{F}(f)(\xi)$ 
\item $\mathcal{F}(x_if(x))=-\frac{1}{2\pi i} \partial_{\xi_i} \mathcal{F}(f)(\xi)$
\end{enumerate}
\end{thm}

\section{Important Inequalities}

\subsection{Sobolev Embedding}

\begin{thm}[Sobolev embedding thm; source: sheet 1]
There exists a constant $C=C(n,s)>0$ such that for every $f\in H^s(\mathbb{R}^n)$ with $s>\frac{n}{2}$ we have
\begin{equation}
\lVert f \rVert_{L^{\infty}(\mathbb{R}^n)} \leq C \lVert f \rVert _{H^s(\mathbb{R}^n)}.
\end{equation}
\end{thm}

\begin{thm}[source: sheet 1]
If $s$ is a non-negative integer, there exists a constant $C=C(s)$ such that for every $f\in H^s(\mathbb{R}^n)$ with $s>\frac{n}{2}$ we have
\begin{equation}
\frac{1}{C} \sum_{\lvert\alpha\rvert\leq s} \lVert \partial ^{\alpha} f \rVert_{L^2(\mathbb{R}^n)} \leq \lVert f \rVert _{H^s(\mathbb{R}^n)} \leq C \sum_{\lvert\alpha\rvert\leq s} \lVert \partial ^{\alpha} f \rVert_{L^2(\mathbb{R}^n)}.
\end{equation}
\end{thm}

\begin{thm}[second Sobolev embedding thm; source: Wiki]
If $n<pk$ and $r+\alpha=k-\frac{n}{p}$ with $\alpha \in (0,1)$, then one has the embedding
\begin{equation}
W^{k,p}(\mathbb{R}^n) \subset C^{r,\alpha}(\mathbb{R}^n).
\end{equation}
\end{thm}


\section{Basic Analysis}

\begin{thm}[Young's Inequality]
Let $p$ and $q$ be positive real numbers such that $\frac{1}{p} + \frac{1}{q} = 1$. Then, for any non-negative $a$ and $b$ it holds
\begin{equation}
ab=\frac{a^p}{p}+\frac{b^q}{q}.
\end{equation}
\end{thm}

\end{document}